\documentclass{article}

\usepackage{graphicx}
\usepackage{verbatim}
\usepackage{subcaption}
\usepackage{amsmath}
\usepackage{hyperref}

\setlength{\parindent}{0cm}  % to remove indentation at the beginning of paragraph.

\title{\LaTeX Lecture 2}
\author{Tareq}

\begin{document}
\maketitle

\section{Figures}

\subsection{Single Figure}
Figures need graphicx package. 

\begin{verbatim}
\usepackage{graphicx}.

\begin{figure}[h] 
\centering
\includegraphics[width=0.3\textwidth]{buetlogo.png}
\caption{Logo of BUET}
\label{fig:logobuet}
\end{figure}
\end{verbatim}

\begin{figure}[h]
	\centering
	\includegraphics[width=0.3\textwidth]{buetlogo.png}
	\caption{Logo of BUET}
	\label{fig:logobuet}
\end{figure}

\subsection{Multiple Figures}

You have to include subcaption package for side by side images.

\begin{verbatim}
\usepackage{subcaption}

\begin{figure}[h]
	\centering
	\begin{subfigure}{0.4\textwidth}
		\centering
		\includegraphics[width=0.8\textwidth]{ubuntulogo.png}
		\caption{Ubuntu}
	\end{subfigure}
	~
	\begin{subfigure}{0.4\textwidth}
		\centering
		\includegraphics[width=0.8\textwidth]{kalilogo.png}
		\caption{Kali Linux}
	\end{subfigure}
	\caption{Linux Distributions}
\end{figure}

\end{verbatim}

\begin{figure}[h]
	\centering
	\begin{subfigure}{0.4\textwidth}
		\centering
		\includegraphics[width=0.8\textwidth]{ubuntulogo.png}
		\caption{Ubuntu}
	\end{subfigure}
	~
	\begin{subfigure}{0.4\textwidth}
		\centering
		\includegraphics[width=0.8\textwidth]{kalilogo.png}
		\caption{Kali Linux}
	\end{subfigure}
	\caption{Linux distributions}
\end{figure}

\section{Equation}

\subsection{Simple Equation}

This is a simple equation: $a + a = 2a$\\
\verb|$a + a = 2a$|\\
This is not math mode: a + a = 2a.

\section{Superscript and Subscript}

Use \verb|_| for subscript, e.g. \verb|$a_n$| shows $a_n$.\\
Use \verb|^| for superscript, e.g. \verb|$a^n$| shows $a^n$.\\
Use \verb|{}| to group more than one characters, e.g. \verb|$a_{in}$| shows $a_{in}$.\\
Also \verb|$^nP_r$| gives us $^nP_r$.

More complex equation, \verb|$$\sum_{i=0}^n a_i$$| shows $$\sum_{i=0}^n a_i$$.

\subsection{Mathematical Environments}

Inline: \verb|$\sum_{i=0}^n a_i$|\\
Inline with displaystyle: \verb|$\displaystyle\sum_{i=0}^n a_i$|\\
Block level: \verb|$$\sum_{i=0}^n a_i$$|\\
Block level with equation number:
\begin{verbatim}
\begin{equation}
\sum_{i=0}^n a_i
\end{equation}
\end{verbatim}
Block level without equation number:
\begin{verbatim}
\begin{equation*}
\sum_{i=0}^n a_i
\end{equation*}
\end{verbatim}

\textbf{Notice the difference in output.}\\
Inline: $\sum_{i=0}^n a_i$\\
Inline with displaystyle: $\displaystyle\sum_{i=0}^n a_i$\\
Block level: $$\sum_{i=0}^n a_i$$\\
Block level with equation number:
\begin{equation}
\sum_{i=0}^n a_i
\end{equation}
Block level without equation number:
\begin{equation*}
\sum_{i=0}^n a_i
\end{equation*}


\subsection{Equation Labeling}

\begin{equation}
\label{eq:sum}
\sum_{i=0}^n a_i
\end{equation}

\begin{verbatim}
\begin{equation}
\label{eq:sum}
\sum_{i=0}^n a_i
\end{equation}
\end{verbatim}

Equation \ref{eq:sum} shows sum of numbers from zero to $n$.


\subsection{Equation Alignment}

\begin{align*}
A & = \frac{\pi r^2}{2} \\
 & = \frac{1}{2} \pi r^2
\end{align*}

\begin{verbatim}
\begin{align*}
A & = \frac{\pi r^2}{2} \\
 & = \frac{1}{2} \pi r^2
\end{align*}
\end{verbatim}


\subsection{Cases}

\begin{equation}
    |x| = 
    \begin{cases}
    x & \textrm{if } x \geq 0\\
    -x & \textrm{if } x < 0\\
    \end{cases}
\end{equation}

\begin{verbatim}
\begin{equation}
    |x| = 
    \begin{cases}
    x & \textrm{if } x \geq 0\\
    -x & \textrm{if } x < 0\\
    \end{cases}
\end{equation}
\end{verbatim}

\subsection{Miscellaneous}

For some of the commands, you need to include \verb|amsmath| package.\\

Comparisons:\\
\verb|$ < \leq $| shows, $ < \leq $\\
\verb|$ > \geq $| shows, $ > \geq $\\

Set operations:\\
\verb|\forall x \in X, \exists y \leq \epsilon|\\
$$ \forall x \in X, \exists y \leq \epsilon $$

\verb|A \cap B, A \cup B|\\
$$ A \cap B, A \cup B $$

Limits and Infinity:\\
\verb|\lim_{x \to \infty} \exp(-x) = 0|\\
$$ \lim_{x \to \infty} \exp(-x) = 0 $$

Fractions:\\
\verb|\frac{a}{b}|\\
$$ \frac{a}{b} $$

Binomials:\\
\verb|\binom{n}{k}|\\
$$ \binom{n}{k} $$

Times:\\
\verb|a \times b|\\
$$ a \times b $$

Root:\\
square root: $ \sqrt{a} $, \verb|\sqrt{a}|\\
$n$th root: $ \sqrt[n]{a} $, \verb|\sqrt[n]{a}|\\

Modular:\\
\verb|a \bmod b, a \pmod b, a \equiv b|\\
$$ a \bmod b, a \pmod b, a \equiv b $$

Integrals:\\
\verb|\int_a^b xdx|\\
$$ \int_a^b xdx $$

Plus minus:\\
\verb|a \pm 5, a \mp 5|\\
$$ a \pm 5, a \mp 5 $$

Trigonometry:\\
\verb|\cos2\theta = \cos^2 - \sin^2|\\
$$\cos2\theta = \cos^2 - \sin^2$$

Derivatives:\\
\verb|\frac{du}{dt} \quad \frac{\partial u}{\partial t}|\\
$$\frac{du}{dt} \quad \frac{\partial u}{\partial t}$$

Custom operator:\\
\verb|\textrm{cos}2\theta = \textrm{cos}^2 - \textrm{sin}^2|\\
$$\textrm{cos}2\theta = \textrm{cos}^2 - \textrm{sin}^2$$

\subsection{Find More Symbols}

Visit \url{http://detexify.kirelabs.org/classify.html} and draw symbol by hand.

\subsection{Autometic Sizing of Parentheses/Braces/Brackets}

\verb|(\frac{a}{b})|\\
$$ (\frac{a}{b}) $$
\\~\\
\verb|\left(\frac{a}{b}\right)|
$$ \left(\frac{a}{b}\right) $$


\subsection{Matrices}

\textbf{matrix}
\begin{verbatim}
\begin{equation}
	\begin{matrix}
		1 & 0 & 0\\
		0 & 1 & 0\\
		0 & 0 & 1\\
	\end{matrix}
\end{equation}
\end{verbatim}

\begin{equation}
	\begin{matrix}
		1 & 0 & 0\\
		0 & 1 & 0\\
		0 & 0 & 1\\
	\end{matrix}
\end{equation}
\\~\\
\textbf{pmatrix}
\begin{verbatim}
\begin{equation}
	\begin{pmatrix}
		1 & 0 & 0\\
		0 & 1 & 0\\
		0 & 0 & 1\\
	\end{pmatrix}
\end{equation}
\end{verbatim}

\begin{equation}
	\begin{pmatrix}
		1 & 0 & 0\\
		0 & 1 & 0\\
		0 & 0 & 1\\
	\end{pmatrix}
\end{equation}
\\~\\
\textbf{bmatrix}
\begin{equation}
	\begin{bmatrix}
		1 & 0 & 0\\
		0 & 1 & 0\\
		0 & 0 & 1\\
	\end{bmatrix}
\end{equation}
\\~\\
Change the matrix environment to $pbBvV$matrix in the following equation and observe the output:


\section{Bibliography}

Process of compiling external bib file (assuming tex file name is doc.tex, it does not matter what the bib file name is):
\begin{verbatim}
pdflatex doc.tex
bibtex doc
pdflatex doc.tex
pdflatex doc.tex
\end{verbatim}

Refer to a work/paper/journal by \verb|\cite{tag}|. You can get BibTex from Google Scholar.

For example, Convolutional Neural Network (CNN) \cite{lecun1998gradient} has been sucessfully applied in various areas of computer vision. Large datasets like ImageNet \cite{deng2009imagenet} can be used in trainning a CNN.

\bibliographystyle{plain}
\bibliography{bibfile}


\end{document}